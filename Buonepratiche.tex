% !TeX spellcheck = it_IT
% !TeX root = UsoMail.tex
\chapter{Buone pratiche}
%\printpartialtoc
\section{Glossario}
\begin{itemize}
	\item Password stringa alfanumerica di almeno otto caratteri
	\item Password provvisoria: Perugia01
	\item Amministratore: gestore del sistema competente per sede
	\item Coordinatore: nelle classi prime corrisponde alla commissione 
	accoglienza
	\item Accesso al sistema: Andare alla pagina di accesso di un qualunque 
	prodotto Google, inserire la mail istituzionale, inserire la password e 
	dare invio
	\item Verifica in due passaggi Il sistema chiede oltre alla password l'uso 
	di un codice che può arrivare anche tramite sms. In questo caso può essere necessario
	l'inserimento di un numero telefonico personale. Tale numero è richiesto d Google e 
	non dall'amministrazione. Al termine della procedura è buona prassi 
	cancellarlo. In questi casi può essere utile l'intervento diretto di un 
	amministratore.
\end{itemize}
\section{Creazione di un'utenza}
Vediamo come dovrebbe essere creata un'utenza scolastica
\begin{enumerate}
	\item La segreteria all'atto dell'iscrizione consegna all'alunno il consenso
	informato e lo allega alla domanda di iscrizione.
	\item La segreteria fornisce le nuove utenze anche in modo massivo.
	\item La segreteria deve indicare la classe dell'alunno
	\item L'amministratore crea l'utenza assegnando una mail provvisoria
	\item L'amministratore assegna l'alunno al suo gruppo classe
	\item L'amministratore consegna al coordinatore l'elenco delle mail
	\item Il coordinatore  consegna le mail alla classe 
	\item La classe procede all'accesso
	\item Il coordinatore comunica all'ufficio tecnico chi ha fatto l'accesso
	\item L'ufficio tecnico invia  a questi, tramite l'email istituzionale, 
	l'invito  per il registro elettronico
\end{enumerate}
Vediamo come dovrebbe essere creata un'utenza docente
\begin{enumerate}
	\item La segreteria fornisce le nuove utenze anche in modo massivo.
	\item La segreteria deve indicare la classe di concorso
	\item L'amministratore crea l'utenza assegnando una mail provvisoria
	\item L'amministratore assegna il docente al suo dipartimento e ai sue 
	gruppi  classe
	\item L'utente provvede all'accesso
	\item L'amministratore  comunica all'ufficio tecnico chi ha fatto l'accesso
	\item L'ufficio tecnico invia  a questi, tramite l'email istituzionale, 
	l'invito  per il registro elettronico
\end{enumerate}
Vediamo come dovrebbe essere creata un'utenza Amministrativa
\begin{enumerate}
	\item La segreteria fornisce le nuove utenze anche in modo massivo.
	\item L'amministratore crea l'utenza assegnando una mail provvisoria
	\item L'amministratore assegna l'amministrativo al suo gruppo
	\item L'utente provvede all'accesso
	\item L'amministratore  comunica all'ufficio tecnico chi ha fatto l'accesso
	\item L'ufficio tecnico invia  a questi, tramite l'email istituzionale, 
	l'invito  per il registro elettronico.
\end{enumerate}
\section{Recupero di un'utenza}
Vediamo come recuperare un'utenza alunno:
\begin{enumerate}
	\item L'alunno comunica la perdita della password al coordinatore
	\item Il coordinatore informa l'amministratore
	\item L'amministratore resetta la password all'utente
	\item L'amministratore assegna una password provvisoria 
	\item L'alunno si connette e immette una nuova password di almeno otto 
	caratteri 
\end{enumerate}
Vediamo come recuperare un'utenza docente:
\begin{enumerate}
	\item Il docente comunica la perdita della password al amministratore
	\item L'amministratore resetta la password all'utente
	\item L'amministratore assegna una password provvisoria 
	\item Il docente si connette e immette una nuova password di almeno otto 
	caratteri 
\end{enumerate}
Vediamo come recuperare un'utenza amministrativa:
\begin{enumerate}
	\item L'amministrativo comunica la perdita della password al amministratore
	\item L'amministratore resetta la password all'utente
	\item L'amministratore assegna una password provvisoria 
	\item L'amministrativo si connette e immette una nuova password di almeno 
	otto caratteri 
\end{enumerate}
\section{Password divulgata}
Password divulgata Google segnala che la password di un utente è compromessa e 
l'account viene sospeso.
\begin{enumerate}
	\item L'amministratore sblocca l'utente
	\item Viene assegnata una nuova mail provvisoria 
	\item Può essere necessaria una verifica in due passaggi
\end{enumerate}
\section{Cancellazione e sospensione}
Il Garante della Privacy è chiaro l'utente deve essere sospeso e 
poi cancellato\footcite{Garante2019}. La procedura è descritta al punto 7 del nostro regolamento\ref{pdf:autworkspace}
