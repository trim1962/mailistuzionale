% !TeX spellcheck = it_IT
% !TeX root = UsoMail.tex
\chapter{Google Workspace}
La MI permette,tramite la scuola, l'accesso ai servizi, forniti da \textenglish{Google}, di \textenglish{Google Workspace}. Tali servizi sono regolati da un contratto sottoscritto dalla scuola~\footcite{Google2020}. La versione in uso dalla scuola è:  \textenglish{Google Workspace for Education Fundamentals}.

\textenglish{Google} suddivide i servizi che fornisce in due gruppi "Servizi principali" e in "Servizi aggiuntivi"

I "Servizi principali" sono  descritti da \textenglish{Google} in riepilogo dei servizi~\footcite{Google2022d} e sono in gran parte riportando la descrizione:
\begin{itemize}
	\item \textenglish{Gmail} è un servizio email basato sul Web che consente a un'organizzazione di gestire il proprio sistema di posta elettronica utilizzando i sistemi di \textenglish{Google}. Permette all'Utente finale di accedere alla propria casella di posta da un browser web supportato, leggere la posta, scrivere, rispondere a messaggi e inoltrarli, cercare nella posta e organizzarla tramite etichette. Offre inoltre filtri antispam e antivirus e consente agli Amministratori di creare regole per la gestione dei messaggi con contenuti specifici e file allegati o di indirizzare i messaggi ad altri server di posta. È possibile configurare regole per gruppo o per Cliente (tutti i domini).
	\item \textenglish{Google Calendar} è un servizio basato sul Web per la gestione di calendari personali, dell'azienda/organizzazione e dei team. Fornisce agli Utenti finali un'interfaccia in cui visualizzare i calendari, pianificare riunioni con gli altri Utenti finali, vedere le informazioni sulla disponibilità degli altri Utenti finali e prenotare stanze e risorse.
	\item "Documenti", "Fogli", "Presentazioni" e "Moduli Google" sono servizi basati sul Web che permettono agli Utenti finali di creare, modificare, condividere, disegnare, esportare, incorporare e lavorare in modo collaborativo su contenuti di documenti, fogli di lavoro, presentazioni e moduli.
	\item "Google Drive" fornisce strumenti basati sul Web che consentono agli Utenti finali di archiviare, trasferire e condividere file, nonché di guardare video.
	\item "Google Groups for Business" è un servizio basato sul Web che consente agli Utenti finali e ai proprietari di siti web di creare e gestire gruppi di collaborazione. Gli Utenti finali possono intrattenere discussioni via email e condividere documenti, calendari, siti e cartelle con i membri di un gruppo. Inoltre, possono visualizzare gli archivi delle discussioni del gruppo ed eseguire ricerche in questi ultimi.
	\item "Google Hangouts", "Google Chat" e "Google Meet" sono servizi basati sul Web che consentono agli Utenti finali di comunicare tra loro in tempo reale. Google Hangouts consente di comunicare in conversazioni a due e di gruppo tramite messaggistica chat e audio e riunioni video di base. Google Chat offre una piattaforma avanzata per la messaggistica via chat e la collaborazione in gruppo che supporta le integrazioni di contenuti con servizi di terze parti selezionati. Google Meet permette di organizzare riunioni video avanzate con un elevato numero di partecipanti, inclusa la possibilità di connettersi o di aggiungere partecipanti a una riunione via telefono.
	\item "Google Jamboard" è un servizio basato sul Web che consente agli Utenti finali di creare, modificare, condividere, disegnare, esportare, incorporare e lavorare in modo collaborativo su contenuti all'interno di un documento.
	\item "Google Sites" consente all'Utente finale di creare un sito mediante uno strumento basato sul Web per poi condividerlo con un gruppo di altri Utenti finali oppure pubblicarlo per tutta l'azienda o esternamente. Il proprietario del sito può scegliere chi può modificare il sito e chi può visualizzarlo.
	\item "Compiti" è un'applicazione per i sistemi di gestione dell'apprendimento che consente agli Utenti finali di distribuire, raccogliere e valutare il lavoro degli studenti.
	\item "Classroom" è un servizio basato sul Web che consente agli Utenti finali di creare e partecipare ai gruppi delle classi. Con Classroom, gli studenti possono visualizzare e consegnare i compiti e ricevere le valutazioni degli insegnanti.
	\item "Sincronizzazione Chrome" è una funzionalità che consente agli Utenti finali di sincronizzare preferiti, cronologia, password e altre impostazioni su tutti i dispositivi su cui hanno eseguito l'accesso a Chrome.
\end{itemize}