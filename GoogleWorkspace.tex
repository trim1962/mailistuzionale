% !TeX spellcheck = it_IT
% !TeX root = UsoMail.tex
\chapter{Google Workspace}
\printpartialtoc
\section{I servizi di Google}
La MI permette,tramite la scuola, l'accesso ai servizi, forniti da \textenglish{Google}, di \textenglish{Google Workspace}. Tali servizi sono regolati da un contratto sottoscritto dalla scuola~\footcite{Google2020}. La versione in uso dalla scuola è:  \textenglish{Google Workspace for Education Fundamentals}.

\textenglish{Google} suddivide i servizi che fornisce in due gruppi "Servizi principali" e in "Servizi aggiuntivi"\footcite{Google2021a}

I "Servizi principali" sono  descritti da \textenglish{Google} in riepilogo dei servizi~\footcite{Google2022d} e sono in gran parte riportando la descrizione:
\begin{itemize}
	\item \textenglish{"Gmail"} è un servizio email basato sul Web che consente a un'organizzazione di gestire il proprio sistema di posta elettronica utilizzando i sistemi di \textenglish{Google}. Permette all'Utente finale di accedere alla propria casella di posta da un browser web supportato, leggere la posta, scrivere, rispondere a messaggi e inoltrarli, cercare nella posta e organizzarla tramite etichette. Offre inoltre filtri antispam e antivirus e consente agli Amministratori di creare regole per la gestione dei messaggi con contenuti specifici e file allegati o di indirizzare i messaggi ad altri server di posta. È possibile configurare regole per gruppo o per Cliente (tutti i domini).
	\item \textenglish{"Google Calendar"} è un servizio basato sul Web per la gestione di calendari personali, dell'azienda/organizzazione e dei team. Fornisce agli Utenti finali un'interfaccia in cui visualizzare i calendari, pianificare riunioni con gli altri Utenti finali, vedere le informazioni sulla disponibilità degli altri Utenti finali e prenotare stanze e risorse.
	\item "Documenti", "Fogli", "Presentazioni" e "Moduli \textenglish{Google}" sono servizi basati sul Web che permettono agli Utenti finali di creare, modificare, condividere, disegnare, esportare, incorporare e lavorare in modo collaborativo su contenuti di documenti, fogli di lavoro, presentazioni e moduli.
	\item \textenglish{"Google Drive"} fornisce strumenti basati sul Web che consentono agli Utenti finali di archiviare, trasferire e condividere file, nonché di guardare video.
	\item \textenglish{"Google Groups for Business"} è un servizio basato sul Web che consente agli Utenti finali e ai proprietari di siti web di creare e gestire gruppi di collaborazione. Gli Utenti finali possono intrattenere discussioni via email e condividere documenti, calendari, siti e cartelle con i membri di un gruppo. Inoltre, possono visualizzare gli archivi delle discussioni del gruppo ed eseguire ricerche in questi ultimi.
	\item  \textenglish{"Google Hangouts"},  \textenglish{"Google Chat"} e   \textenglish{"Google Meet"}sono servizi basati sul Web che consentono agli Utenti finali di comunicare tra loro in tempo reale. 
	
	\textenglish{"Google Hangouts"} consente di comunicare in conversazioni a due e di gruppo tramite messaggistica chat e audio e riunioni video di base. 
	
	\textenglish{"Google Chat"} offre una piattaforma avanzata per la messaggistica via chat e la collaborazione in gruppo che supporta le integrazioni di contenuti con servizi di terze parti selezionati. 
	
	\textenglish{"Google Meet"} permette di organizzare riunioni video avanzate con un elevato numero di partecipanti, inclusa la possibilità di connettersi o di aggiungere partecipanti a una riunione via telefono.
	\item \textenglish{"Google Jamboard"} è un servizio basato sul Web che consente agli Utenti finali di creare, modificare, condividere, disegnare, esportare, incorporare e lavorare in modo collaborativo su contenuti all'interno di un documento.
	\item \textenglish{"Google Sites"} consente all'Utente finale di creare un sito mediante uno strumento basato sul Web per poi condividerlo con un gruppo di altri Utenti finali oppure pubblicarlo per tutta l'azienda o esternamente. Il proprietario del sito può scegliere chi può modificare il sito e chi può visualizzarlo.
	\item "Compiti" è un'applicazione per i sistemi di gestione dell'apprendimento che consente agli Utenti finali di distribuire, raccogliere e valutare il lavoro degli studenti.
	\item \textenglish{"Classroom"} è un servizio basato sul Web che consente agli Utenti finali di creare e partecipare ai gruppi delle classi. Con \textenglish{"Classroom"}, gli studenti possono visualizzare e consegnare i compiti e ricevere le valutazioni degli insegnanti.
	\item "Sincronizzazione  \textenglish{Chrome}" è una funzionalità che consente agli Utenti finali di sincronizzare preferiti, cronologia, password e altre impostazioni su tutti i dispositivi su cui hanno eseguito l'accesso a \textenglish{Chrome}.
\end{itemize}
I "Servizi aggiuntivi" sono dei servizi non compresi nell'elenco precedente. Quindi sono aggiuntivi per esempio:
\begin{itemize}
	\item \textenglish{"Google Play"}
	\item \textenglish{"YouTube"}
	\item \textenglish{"Google Maps"}
	\item \textenglish{"Blogger"}
\end{itemize}
I "Servizi aggiuntivi" richiedono che venga richiesta ai Genitori o tutori una autorizzazione aggiuntiva~\footcite{Google2022e} 
\section{I dati raccolti da Google}
\textenglish{Google} raccogli i dati in vari modi\footcite{Google2022a}. La scuola creando un utenza Workspace fornisce a Google una mail, la password e un nome e un cognome. A questo nucleo minimo di informazioni è possibile aggiungere una mail secondaria, un numero di telefono per il recupero password etc..

Google oltre ai dati forniti alla creazione dell'utenza, raccoglie i dati prodotti dagli utenti con gli strumenti forniti. 