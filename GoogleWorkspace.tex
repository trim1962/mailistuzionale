% !TeX spellcheck = it_IT
% !TeX root = UsoMail.tex
\chapter{Google Workspace}
%\printpartialtoc
\section{I servizi di Google}
La MI permette,tramite la scuola, l'accesso ai servizi, forniti da \textenglish{Google}, di \textenglish{Google Workspace}. Tali servizi sono regolati da un contratto sottoscritto dalla scuola~\footcite{Google2020}. La versione in uso dalla scuola è:  \textenglish{Google Workspace for Education Fundamentals}.

\textenglish{Google} suddivide i servizi che fornisce in due gruppi "Servizi principali" e in "Servizi aggiuntivi"\footcite{Google2021a}

I "Servizi principali" sono  descritti da \textenglish{Google} in riepilogo dei servizi~\footcite{Google2022d} e sono in gran parte riportando la descrizione:
\begin{itemize}
	\item \textenglish{"Gmail"} è un servizio email basato sul Web che consente a un'organizzazione di gestire il proprio sistema di posta elettronica utilizzando i sistemi di \textenglish{Google}. Permette all'Utente finale di accedere alla propria casella di posta da un browser web supportato, leggere la posta, scrivere, rispondere a messaggi e inoltrarli, cercare nella posta e organizzarla tramite etichette. Offre inoltre filtri antispam e antivirus e consente agli Amministratori di creare regole per la gestione dei messaggi con contenuti specifici e file allegati o di indirizzare i messaggi ad altri server di posta. È possibile configurare regole per gruppo o per Cliente (tutti i domini).
	\item \textenglish{"Google Calendar"} è un servizio basato sul Web per la gestione di calendari personali, dell'azienda/organizzazione e dei team. Fornisce agli Utenti finali un'interfaccia in cui visualizzare i calendari, pianificare riunioni con gli altri Utenti finali, vedere le informazioni sulla disponibilità degli altri Utenti finali e prenotare stanze e risorse.
	\item "Documenti", "Fogli", "Presentazioni" e "Moduli \textenglish{Google}" sono servizi basati sul Web che permettono agli Utenti finali di creare, modificare, condividere, disegnare, esportare, incorporare e lavorare in modo collaborativo su contenuti di documenti, fogli di lavoro, presentazioni e moduli.
	\item \textenglish{"Google Drive"} fornisce strumenti basati sul Web che consentono agli Utenti finali di archiviare, trasferire e condividere file, nonché di guardare video.
	\item \textenglish{"Google Groups for Business"} è un servizio basato sul Web che consente agli Utenti finali e ai proprietari di siti web di creare e gestire gruppi di collaborazione. Gli Utenti finali possono intrattenere discussioni via email e condividere documenti, calendari, siti e cartelle con i membri di un gruppo. Inoltre, possono visualizzare gli archivi delle discussioni del gruppo ed eseguire ricerche in questi ultimi.
	\item  \textenglish{"Google Hangouts"},  \textenglish{"Google Chat"} e   \textenglish{"Google Meet"}sono servizi basati sul Web che consentono agli Utenti finali di comunicare tra loro in tempo reale. 
	
	\textenglish{"Google Hangouts"} consente di comunicare in conversazioni a due e di gruppo tramite messaggistica chat e audio e riunioni video di base. 
	
	\textenglish{"Google Chat"} offre una piattaforma avanzata per la messaggistica via chat e la collaborazione in gruppo che supporta le integrazioni di contenuti con servizi di terze parti selezionati. 
	
	\textenglish{"Google Meet"} permette di organizzare riunioni video avanzate con un elevato numero di partecipanti, inclusa la possibilità di connettersi o di aggiungere partecipanti a una riunione via telefono.
	\item \textenglish{"Google Jamboard"} è un servizio basato sul Web che consente agli Utenti finali di creare, modificare, condividere, disegnare, esportare, incorporare e lavorare in modo collaborativo su contenuti all'interno di un documento.
	\item \textenglish{"Google Sites"} consente all'Utente finale di creare un sito mediante uno strumento basato sul Web per poi condividerlo con un gruppo di altri Utenti finali oppure pubblicarlo per tutta l'azienda o esternamente. Il proprietario del sito può scegliere chi può modificare il sito e chi può visualizzarlo.
	\item "Compiti" è un'applicazione per i sistemi di gestione dell'apprendimento che consente agli Utenti finali di distribuire, raccogliere e valutare il lavoro degli studenti.
	\item \textenglish{"Classroom"}\index{Classroom} è un servizio basato sul Web che consente agli Utenti finali di creare e partecipare ai gruppi delle classi. Con \textenglish{"Classroom"}, gli studenti possono visualizzare e consegnare i compiti e ricevere le valutazioni degli insegnanti.
	\item "Sincronizzazione  \textenglish{Chrome}" è una funzionalità che consente agli Utenti finali di sincronizzare preferiti, cronologia, password e altre impostazioni su tutti i dispositivi su cui hanno eseguito l'accesso a \textenglish{Chrome}.
\end{itemize}
I "Servizi aggiuntivi" sono dei servizi non compresi nell'elenco precedente. Quindi sono aggiuntivi per esempio:
\begin{itemize}
	\item \textenglish{"Google Play"}
	\item \textenglish{"YouTube"}
	\item \textenglish{"Google Maps"}
	\item \textenglish{"Blogger"}
\end{itemize}
I "Servizi aggiuntivi" richiedono che venga richiesta ai Genitori o tutori una autorizzazione aggiuntiva~\footcite{Google2022e} 
\section{I dati raccolti da Google}
\textenglish{Google} raccogli i dati in vari modi\footcite{Google2022a}. La scuola creando un utenza Workspace fornisce a Google una mail, la password,  nome e  cognome. A questo nucleo minimo di informazioni è possibile aggiungere una mail secondaria, un numero di telefono per il recupero password etc..

Google oltre ai dati forniti alla creazione dell'utenza, raccoglie i dati prodotti dagli utenti tramite gli strumenti forniti. 

Ricapitolando Google raccoglie\footcite{Google2022a}
\begin{itemize}
	\item I dati dell'account, incluse informazioni come nome e indirizzo email.
	\item Impostazioni, app, browser e dispositivi. Raccogliamo informazioni sulle  impostazioni e su, app, browser e dispositivi che utilizzati per accedere  servizi Google. Queste informazioni includono il tipo di browser e di dispositivo, le impostazioni, gli identificatori univoci, il sistema operativo, le informazioni sulla rete mobile e il numero di versione delle applicazioni. Raccoglie inoltre informazioni sull'interazione tra le tue app, browser e dispositivi e i servizi forniti da Google, inclusi l'indirizzo IP, i report sugli arresti anomali, l'attività di sistema e la data e l'ora della tua richiesta.
	\item Informazioni sulla posizione. Goggle raccoglie informazioni sulla posizione ricavate tramite varie tecnologie, tra cui l'indirizzo IP e il GPS.
	\item Comunicazioni dirette da parte utente. Registriamo le comunicazioni quando l'utente o il suo amministratore forniscono feedback, fanno domande o chiedono aiuto all'assistenza tecnica.
\end{itemize}
\begin{center}
	\begin{tabular}{ll}
	\toprule
	Servizi&Stato del servizio\\
	\midrule 
Account del brand&
ATTIVO per tutti\\
Applicazioni sperimentali&
ATTIVO per tutti\\
Applied Digital Skills&
ATTIVO per tutti\\
App Maker&
ATTIVO per tutti\\
AppSheet&
ATTIVO per tutti\\
Backup di applicazioni di terze parti&
ATTIVO per tutti\\
Blogger&
ATTIVO per tutti\\
Campaign Manager&
NON ATTIVO\\
Centro partner di Google Play Libri&
NON ATTIVO\\
Chrome Canvas&
ATTIVO per tutti\\
Chrome Remote Desktop&
ATTIVO per tutti\\
Chrome Web Store&
ATTIVO per tutti\\
Colab&
ATTIVO per tutti\\
CS First&
NON ATTIVO\\
Currents&
ATTIVO per tutti\\
FeedBurner&
ATTIVO per tutti\\
Google Ad Manager&
NON ATTIVO\\
Google Ads&
NON ATTIVO\\
Google AdSense&
NON ATTIVO\\
Google Alert&
ATTIVO per tutti\\
Google Analytics&
NON ATTIVO\\
Google Arts and Culture&
ATTIVO per tutti\\
Google Cloud Platform&
ATTIVO per tutti\\
Google Cloud Print&
ATTIVO per tutti\\
Google Developers&
ATTIVO per tutti\\
Google Domains&
NON ATTIVO\\
Google Earth&
ATTIVO per tutti\\
Google Fi&
NON ATTIVO\\
Google Foto&
ATTIVO per tutti\\
Google Gruppi&
ATTIVO per tutti\\
Google Libri&
ATTIVO per tutti\\
Google Maps&
ATTIVO per tutti\\
Google My Business&
NON ATTIVO\\
Google My Maps&
ATTIVO per tutti\\
Google News&
ATTIVO per tutti\\
Google Pay&
NON ATTIVO\\
Google Play&
ATTIVO per tutti\\
Google Play Console&
ATTIVO per tutti\\
Google Public Data&
ATTIVO per tutti\\
Google Search Console&
ATTIVO per tutti\\
Google Segnalibri&
ATTIVO per tutti\\
Google Traduttore&
ATTIVO per tutti\\
Google Translator Toolkit&
NON ATTIVO\\
Google Viaggi&
NON ATTIVO\\
Google Voice&
ATTIVO per tutti\\
Location History&
NON ATTIVO\\
Looker Studio&
ATTIVO per tutti\\
Material Gallery&
ATTIVO per tutti\\
Merchant Center&
NON ATTIVO\\
Messaggi Google&
ATTIVO per tutti\\
\bottomrule
\end{tabular}
\captionof{table}{Servizi Google aggiuntivi}
\end{center}
\section{Obblighi contrattuali}
Siamo tenuti a chiedere un'autorizzazione che viene richiesta espressamente da Google\footcite{Google2020}. Il contratto prevede quanto segue:

3.5 Prodotti Aggiuntivi. Google mette a disposizione del Cliente e dei suoi Utenti Finali alcuni Prodotti Aggiuntivi opzionali. L’utilizzo dei Prodotti Aggiuntivi è soggetto ai Termini Aggiuntivi di Prodotto. Il Cliente potrà attivare o disattivare i Prodotti Aggiuntivi in qualsiasi momento attraverso l’Admin Console. Il Cliente otterrà il consenso dei genitori per la raccolta e l’uso dei dati personali nei Prodotti Aggiuntivi che il Cliente intenda abilitare, prima di consentire agli Utenti Finali di età inferiore ai 18 anni di accedere a o utilizzare tali Prodotti Aggiuntivi.
