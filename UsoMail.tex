% !TeX document-id = {b30d14d7-2607-4df4-a1ff-fee288478bc5}
% !TeX spellcheck = it_IT
% !TeX root = UsoMail.tex

\documentclass[structure=book,  
pagelayout=standard,defaultfont=cochineal,partialtoc]{suftesi}
\usepackage{fontspec}
\usepackage[italian]{babel}
\title{Le utenze di Workspace}
\author{Claudio Duchi}
\date{13 dicembre 2022}
\newcommand{\printmail}[1]{\textbf{#1}}
\usepackage[style=italian]{csquotes}
\usepackage[%
style=philosophy-modern,
backend=biber]{biblatex}
\addbibresource{UsoMail.bib}
\partialtocsize{\footnotesize}
\partialtocsecfont{\bfseries\itshape}
\partialtocsubsecfont{\itshape}
\partialtocseclabel{\bfseries}
\partialtocbeforecode{\hrule\smallskip\textbf{Contents}\smallskip}
\partialtocaftercode{\smallskip\hrule}
\begin{document}
\maketitle
\tableofcontents
\chapter{La mail istituzionale}
\printpartialtoc
\section{Definizioni}
La scuola fornisce ad ogni utente una mail istituzionale che permette 
la comunicazione fra gli utenti interni e/o entità esterne alla scuola.
Ogni mail viene rilasciata utilizzando il seguente formato:
\begin{itemize}
	\item \printmail{nome.cognome.doc@iisperugia.net} se l'utente è un docente
	\item \printmail{nome.cognome.stu@iisperugia.net} se l'utente è uno studente
	\item \printmail{nome.cognome.tir@iisperugia.net} se l'utente è un 
	tirocinante
	\item \printmail{nome.cognome.ata@iisperugia.net} se l'utente è un 
	assistente 
	amministrativo.
\end{itemize}
 
Ad ogni mail è associata una password di almeno otto caratteri che permette 
l'accesso ai servizi di Google Workspace altrimenti noto come G Suite.

\section{Regole di creazione o buone prassi} 
La mail viene rilasciata dagli amministrazioni del sistema secondo le seguenti 
regole
Se l'utente ha un solo cognome e un solo nome viene creato
\begin{center}
\printmail{nome.cognome.doc@iisperugia.net}
\end{center}
esempio: Mario Rossi, docente avremo:
\begin{center}
	\printmail{mario.rossi.doc@iisperugia.net}
\end{center}
Se una persona ha più di un nome o di un cognome si usa il primo nome e il  
primo cognome: Carlo Mario Bianchi Rossi studente diventa
\begin{center}
	\printmail{carlo.bianchi.stu@iisperugia.net}
\end{center}
In caso di omonimia viene aggiunto un numero dopo il cognome.
\section{Mail istituzionale e Privacy}
Partiamo da qualche osservazione: la mail è inviolabile,  la mail personale è 
un dato sensibile la 
mail istituzionale è una mail di lavoro.  Anche questa è un dato sensibile  
dato che ha la forma nome.cognome. Per ovviare a ciò sono  
disponibili delle mail alias~\footcite{Garante2007} che nascondono gli utenti 
che  hanno rapporti 
con l'esterno dell'organizzazione. Il Garante~\footcite{Garante2007} chiede che 
venga definito un disciplinare noto agli utenti. Quindi deve essere chiaro che:
la scuola fornendo questo servizio non può, salvo casi eccezionali, utilizzare 
la mail personale dell'utente uso che poi dovrebbe essere autorizzato. 

L'utente non può a sua volta utilizzare la mail istituzionale per  usi  
personali. All'inizio del rapporto l'utente dovrebbe essere informato di questo 
in modo da non potare a fraintendimenti in futuro.

L'utente deve conoscere il percorso di vita della sua utenza, creazione, uso, 
recupero eventuale di dati, 
cancellazione~\footcite{Garante2007}~\footcite{Garante2019}. L'utente deve 
essere consapevole che 
può recuperare i propri dati trasferendo l'utenza. 
\chapter{Buone pratiche}
\printpartialtoc
\section{Glossario}
\begin{itemize}
	\item Password stringa alfanumerica di almeno otto caratteri
	\item Password provvisoria: Perugia01
	\item Amministratore: gestore del sistema competente per sede
	\item Coordinatore: nelle classi prime corrisponde alla commissione 
	accoglienza
	\item Accesso al sistema: Andare alla pagina di accesso di un qualunque 
	prodotto Google, inserire la mail istituzionale, inserire la password e 
	dare invio
\end{itemize}
\section{Creazione di un'utenza}
Vediamo come dovrebbe essere creata un'utenza scolastica
\begin{enumerate}
	\item La segreteria all'atto dell'iscrizione consegna all'alunno il consenso
	informato e lo allega alla domanda di iscrizione.
	\item La segreteria fornisce le nuove utenze anche in modo massivo.
	\item La segreteria deve indicare la classe dell'alunno
	\item L'amministratore crea l'utenza assegnando una mail provvisoria
	\item L'amministratore assegna l'alunno al suo gruppo classe
	\item L'amministratore consegna al coordinatore l'elenco delle mail
	\item Il coordinatore  consegna le mail alla classe 
	\item La classe procede all'accesso
	\item Il coordinatore comunica all'ufficio tecnico chi ha fatto l'accesso
	\item L'ufficio tecnico invia  a questi, tramite l'email istituzionale, 
	l'invito  per il registro elettronico
\end{enumerate}
Vediamo come dovrebbe essere creata un'utenza docente
\begin{enumerate}
	\item La segreteria fornisce le nuove utenze anche in modo massivo.
	\item La segreteria deve indicare la classe di concorso
	\item L'amministratore crea l'utenza assegnando una mail provvisoria
	\item L'amministratore assegna il docente al suo dipartimento e ai sue 
	gruppi  classe
	\item L'utente provvede all'accesso
	\item L'amministratore  comunica all'ufficio tecnico chi ha fatto l'accesso
	\item L'ufficio tecnico invia  a questi, tramite l'email istituzionale, 
	l'invito  per il registro elettronico
\end{enumerate}
Vediamo come dovrebbe essere creata un'utenza Amministrativa
\begin{enumerate}
	\item La segreteria fornisce le nuove utenze anche in modo massivo.
	\item L'amministratore crea l'utenza assegnando una mail provvisoria
	\item L'amministratore assegna l'amministrativo al suo gruppo
	\item L'utente provvede all'accesso
	\item L'amministratore  comunica all'ufficio tecnico chi ha fatto l'accesso
	\item L'ufficio tecnico invia  a questi, tramite l'email istituzionale, 
	l'invito  per il registro elettronico.
\end{enumerate}
\section{Recupero di un'utenza}
Vediamo come recuperare un'utenza alunno:
\begin{enumerate}
	\item L'alunno comunica la perdita della password al coordinatore
	\item Il coordinatore informa all'amministratore
	\item L'amministratore resetta la password all'utente
	\item L'amministratore assegna una password provvisoria 
	\item L'alunno si connette e immette una nuova password di almeno otto 
	caratteri 
\end{enumerate}
Vediamo come recuperare un'utenza docente:
\begin{enumerate}
	\item Il docente comunica la perdita della password al amministratore
	\item Il amministratore informa all'amministratore
	\item L'amministratore resetta la password all'utente
	\item L'amministratore assegna una password provvisoria 
	\item Il docente si connette e immette una nuova password di almeno otto 
	caratteri 
\end{enumerate}
Vediamo come recuperare un'utenza amministrativa:
\begin{enumerate}
	\item L'amministrativa comunica la perdita della password al amministratore
	\item Il amministratore informa all'amministratore
	\item L'amministratore resetta la password all'utente
	\item L'amministratore assegna una password provvisoria 
	\item L'amministrativa si connette e immette una nuova password di almeno 
	otto 
	caratteri 
\end{enumerate}
Password divulgata Google segnala che la password di un utente è compromessa e 
l'account viene sospeso.
\begin{enumerate}
	\item L'amministratore sblocca l'utente
	\item Viene assegnata una nuova mail provvisoria 
\end{enumerate}
\nocite{*}
\printbibliography
\end{document}