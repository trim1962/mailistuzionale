% !TeX root = UsoMail.tex
\chapter{Massive upgrade}
La console di amministrazione permette di inserire e/o modificare molti utenti contemporaneamente. Queste modifiche avvengono tramite un file csv\index{CSV} che deve essere formattato in maniera idonea.
\section{Come ottenere un file CSV}
Per ottenere un file csv idoneo si procede da:
\begin{enumerate}
	\item Amministrazione
	\item Utenti
	\item Gestione
	\item Scarica
	\begin{enumerate}
		\item Seleziona Colonne: Tutte le colonne
		\item Seleziona un formato: Valori separati da virgole (.csv)
	\end{enumerate}
	\item Premere Scarica
	\item In le tue attività: Il file CSV con le informazioni degli utenti è pronto per essere scaricato.
	\item Premere Scarica CSV
	\item Aprire il file, si consiglia Calc di Libre Office
\end{enumerate}
\section{Nuova unità organizzativa}
\begin{table}
\begin{widebox}
\begin{tabular}{llllll}
First Name & Last Name & Email  & Password  & Org Unit Path  & New Status  \\ 
{[Required]} & [Required] & [Required] & [Required] & [Required] & [UPLOAD ONLY] \\ 
Nome & Cognome & nome.cognome@iisperugia.net & **** & /Scarto & Suspended \\  
\end{tabular}

\end{widebox}
\caption{Modifica dell'Unità organizzativa e dello status}
\label{tab:spostnewstat}
\end{table}
