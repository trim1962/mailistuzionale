% !TeX root = UsoMail.tex
\chapter{Massive upgrade}
La console di amministrazione permette di inserire e/o modificare molti utenti contemporaneamente. Queste modifiche avvengono tramite un file CSV\cite{Google2023h}\index{CSV } che deve essere formattato in maniera idonea.
\section{Come ottenere un file CSV}
Per ottenere un file csv idoneo si procede da:
\begin{enumerate}
	\item Amministrazione
	\item Utenti
	\item Gestione
	\item Scarica
	\begin{enumerate}
		\item Seleziona Colonne: Tutte le colonne
		\item Seleziona un formato: Valori separati da virgole (.csv)
	\end{enumerate}
	\item Premere Scarica
	\item In le tue attività: Il file CSV con le informazioni degli utenti è pronto per essere scaricato.
	\item Premere Scarica CSV
	\item Aprire il file, si consiglia Calc\index{Calc} di LibreOffice\index{LibreOffice}
\end{enumerate}

Il file ottenuto ha 37 colonne ma non bisogna spaventarsi. Alcune  sono sempre richieste [Required], altre sono informative, altre sono facoltative [UPLOAD ONLY]. 

Un grosso problema è che al caricamento non viene fatto un controllo preventivo e se vi sono omonimie il sistema non avvisa limitandosi a sovrascrivere. Con un file CSV possiamo caricare fino a 150000 utenti. La password\index{Password} deve essere almeno di otto caratteri.

Terminato il documento si salva in formato CSV e si ritorna in Amministrazione e si carica. Attenzione che viene salvato solo il foglio corrente.
\section{Vari esempi}
\subsection{Sospendere e spostare}
Se dobbiamo sospendere e spostare degli utenti creiamo una tabella come la~\cref{tab:spostnewstat}

Per farlo procediamo in questo modo. Scarichiamo il file CSV, eliminiamo le colonne in eccesso, modifichiamo il Org Unit Path e nella colonna New Status inseriamo Suspended. Salviamo il file in formato CSV apriamo la consolle e da gestione utenti carichiamo dando conferma.
\begin{table}%[!h]
	\centering
	\begin{tabular}{l}
		%\setlength{\arrayrulewidth}{2pt}
		\begin{tabular}{@{}|U{4mm}|U{16mm}|U{16mm}|U{20mm}|U{16mm}|U{16mm}|U{20mm}| }%|C{6mm}|*5{C{22.6mm}|}@{}}
		\hline\rowcolor[gray]{.9}
		&A			&B			&C&D&E&F\tabularnewline		
	\end{tabular}\\
	\begin{tabular}{@{}|>{\columncolor[gray]{.9}}M{4mm}|M{16mm}|M{16mm}|M{20mm} |M{16mm}|M{16mm}|M{20mm}|@{}}
		% oppure
		\hline 1 &First Name [Required] & Last Name  [Required]  & Email  [Required]&Password [Required]&Org Unit Path [Required]& New Status [UPLOAD ONLY] \tabularnewline 
		\hline 2 & nome & cognome & nome.cognome@iisperugia.net &****&\textbackslash Scarto& Suspended \tabularnewline
		\hline 
	\end{tabular}
\end{tabular}
\caption{Modifica dell'Unità organizzativa e dello status}
\label{tab:spostnewstat}
\end{table}
\todo{Aggiungere esempio creazione classe}
\todo{Aggiungere cosa fare a fine anno}