% !TeX spellcheck = it_IT
% !TeX root = UsoMail.tex
\chapter{Ruoli amministrativi predefiniti}
Google ha predisposto dei ruoli predefiniti per gestire Workspace for Education\footcite{Google2023b}. Non tutti i ruoli sono stati assegnati.
\section{Super amministratore}
Ha accesso a tutte le funzioni della Console di amministrazione e dell'API amministrativa; può inoltre gestire ogni aspetto degli account dell'organizzazione.

I super amministratori\index{Amministratore!Super amministratore} hanno inoltre diritti di accesso completi per i calendari di tutti gli utenti e per i dettagli degli eventi. Al momento vi sono solo tre super amministratori.

Solo i super amministratori possono…
\begin{itemize}
\item Creare e assegnare ruoli di amministratore.
\item Gestire altri super amministratori e utenti con delega di amministratore, inclusa la modifica delle password.
\item Trasferire la proprietà dei file durante il processo di eliminazione utente.
\item Accettare i Termini di servizio di un prodotto.
\item Invitare gli account utente non gestiti a diventare account utente gestiti di Google Workspace.
\item Ripristinare gli utenti eliminati.
\item Consentire agli utenti di attivare la verifica in due passaggi.
\item Installare app del Google Workspace Marketplace.
\item Gestire i controlli del livello di accesso alle risorse di Google Calendar.
\item Utilizzare il servizio di migrazione dei dati.
\item Concedere una delega a livello di dominio/gestire l'accesso dei client API.
\item Configurare Google come provider di identità SAML e aggiungere/modificare le app SAML.
\end{itemize}
\section{Amministratore di Gruppi}
Ha il controllo completo sulle attività dei gruppi Google\index{Amministratore!Amministratore di Gruppi} nella Console di amministrazione. Questo amministratore può eseguire le attività seguenti, sia tramite la Console di amministrazione sia tramite l'API amministrativa:
\begin{itemize}
\item Visualizzare i profili utente e la struttura organizzativa.
\item Creare nuovi gruppi nella Console di amministrazione.
\item Gestire i membri dei gruppi creati nella Console di amministrazione.
\item Gestire le impostazioni di accesso del gruppo.
\item Eliminare gruppi dalla Console di amministrazione.
\item Visualizzare le unità organizzative.
\end{itemize}
Ne sono attivi due.
\section{Amministratore gestione utenti}
Può eseguire qualsiasi operazione sugli utenti non amministratori. Questo amministratore\index{Amministratore!Amministratore gestione utenti} può eseguire le attività seguenti, sia tramite la Console di amministrazione sia tramite l'API amministrativa:
\begin{itemize}
\item Visualizzare i profili utente e la struttura organizzativa.
	\item Visualizzare le unità organizzative.
	\item Creare ed eliminare gli account utente. 
	\item Rinominare gli utenti e cambiare le password. 
	\item Gestire le singole impostazioni di sicurezza di un utente. 
	\item Eseguire altre operazioni di gestione degli utenti.
\end{itemize}
Ne sono attivi due.
\section{Amministratore Help Desk}
Può reimpostare le password per gli utenti che non sono amministratori, sia nella Console di amministrazione\index{Amministratore!Amministratore Help Desk} sia tramite l'API amministrativa, e visualizzare i profili utente e la struttura organizzativa. Questo amministratore può visualizzare le unità organizzative.

Non è stato attivato.
\section{Amministratore di servizi}
Può gestire\index{Amministratore!Amministratore di servizi} alcuni dispositivi e impostazioni di servizi e dispositivi nella Console di amministrazione, tra cui Calendar, Google Drive e Documenti. Questo amministratore può:
\begin{itemize}
\item Attivare o disattivare i servizi.
\item Modificare le impostazioni e le autorizzazioni relative a un servizio.
\item Creare, modificare ed eliminare le risorse di Calendar\index{Calendar}
Nota: gli utenti che hanno il ruolo Amministratore di servizi non possono modificare le impostazioni di condivisione delle risorse di Calendar.
\item Gestire dispositivi Chrome e mobili presenti nella Console di amministrazione.
\item Visualizzare le unità organizzative.
\item Utilizzare il Centro avvisi (accesso completo).
\end{itemize}
Non è stato attivato.
\section{Amministratore Spazio di archiviazione}
Può utilizzare\index{Amministratore!Amministratore Spazio di archiviazione} la pagina Spazio di archiviazione della Console di amministrazione. Questo amministratore può:
\begin{itemize}
\item Visualizzare l'utilizzo dello spazio di archiviazione dell'organizzazione.
\item Visualizzare gli utenti e i Drive condivisi che utilizzano la maggiore quantità di spazio di archiviazione.
\item Impostare i limiti di spazio di archiviazione.
\item Aprire il report sull'Account, la directory degli utenti e l'elenco dei Drive condivisi.
\end{itemize}

Questo ruolo concede anche l'accesso completo alle impostazioni Report e Drive.

Non è stato attivato.