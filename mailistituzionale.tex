% !TeX spellcheck = it_IT
% !TeX root = UsoMail.tex
\chapter{La mail istituzionale}
\printpartialtoc
\section{Definizioni}
La scuola fornisce ad ogni utente una mail istituzionale (MI) che permette 
la comunicazione fra gli utenti interni e/o entità esterne alla scuola.
Ogni mail viene rilasciata utilizzando il seguente formato:
\begin{itemize}
	\item \printmail{nome.cognome.doc@iisperugia.net} se l'utente è un docente
	\item \printmail{nome.cognome.stu@iisperugia.net} se l'utente è uno studente
	\item \printmail{nome.cognome.tir@iisperugia.net} se l'utente è un 
	tirocinante
	\item \printmail{nome.cognome.ata@iisperugia.net} se l'utente è un 
	assistente 
	amministrativo.
\end{itemize}

Ad ogni mail è associata una password di almeno otto caratteri che permette 
l'accesso ai servizi di Google Workspace altrimenti noti come G Suite.

\section{Regole di creazione o buone prassi} 
La mail viene rilasciata dagli amministrazioni del sistema secondo le seguenti 
regole
Se l'utente ha un solo cognome e un solo nome viene creato
\begin{center}
	\printmail{nome.cognome.doc@iisperugia.net}
\end{center}
esempio: Mario Rossi, docente avremo:
\begin{center}
	\printmail{mario.rossi.doc@iisperugia.net}
\end{center}
Se una persona ha più di un nome o di un cognome si usa il primo nome e il  
primo cognome: Carlo Mario Bianchi Rossi studente diventa
\begin{center}
	\printmail{carlo.bianchi.stu@iisperugia.net}
\end{center}
In caso di omonimia viene aggiunto un numero dopo il cognome.
\section{Mail istituzionale e Privacy}
Partiamo da qualche osservazione: la mail è inviolabile, la 
MI è uno strumento di lavoro. L'utente deve impegnarsi a controllarla perché permette la comunicazione fra lui e la scuola. La mail è 
un dato sensibile. Per questo sono quindi 
disponibili delle mail alias~\footcite{Garante2007} che nascondono l'identità degli utenti 
che  hanno rapporti 
con l'esterno dell'organizzazione. 

Il Garante~\footcite{Garante2007} chiede che 
venga definito un disciplinare noto agli utenti. Quindi deve essere chiaro che:
la scuola fornendo questo servizio non può, salvo casi eccezionali, utilizzare 
la mail personale dell'utente.

L'utente non può a sua volta utilizzare la mail istituzionale per  usi  
personali. La mail Istituzionale  All'inizio del rapporto l'utente dovrebbe essere informato di questo 
in modo da non portare a fraintendimenti nel futuro.

L'utente deve conoscere il percorso di vita della sua utenza, creazione, uso, 
recupero eventuale di dati, 
cancellazione~\xfootnote{\cite{Garante2007}}~\footcite{Garante2019}. L'utente deve 
essere consapevole che 
può recuperare i propri dati trasferendo l'utenza. 