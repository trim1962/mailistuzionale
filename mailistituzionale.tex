% !TeX spellcheck = it_IT
% !TeX root = UsoMail.tex
\chapter{La mail istituzione}
%\printpartialtoc
\section{Definizioni}
La scuola fornisce ad ogni utente una mail istituzionale\index{Mail!istituzionale} (MI) che permette 
la comunicazione fra gli utenti interni e/o entità esterne alla scuola.

Ogni mail viene rilasciata utilizzando il seguente schema\index{Mail!istituzionale!schema}:
\begin{itemize}
	\item \printmail{nome.cognome.doc@iisperugia.net} se l'utente è un docente
	\item \printmail{nome.cognome.stu@iisperugia.net} se l'utente è uno studente
	\item \printmail{nome.cognome.tir@iisperugia.net} se l'utente è un 
	tirocinante
	\item \printmail{nome.cognome.ata@iisperugia.net} se l'utente è un 
	assistente 
	amministrativo.
\end{itemize}

A una mail corrisponde una password di almeno otto caratteri che permettono
l'accesso ai servizi di  \textenglish{Google Workspace} altrimenti noto come \textenglish{G Suite}.

Il sistema è gestito da amministratori che possono: creare, sospendere, cancellare un'utenza. Possono inoltre, limitare l'uso del sistema a utenti o gruppi di utenze, non possono leggere il contenuto della MI ma controllare se una mail è stata spedita e a chi. Gli amministratori inoltre possono verificare se un utente è entrato nel sistema e leggere i file log del sistema.

\section{Regole di creazione} 
La mail viene rilasciata dagli amministratori del sistema utilizzando le seguenti 
regole:

Se l'utente ha un solo cognome e un solo nome viene creato
\begin{center}
	\printmail{nome.cognome.doc@iisperugia.net}
\end{center}
esempio: Mario Rossi, docente avremo:
\begin{center}
	\printmail{mario.rossi.doc@iisperugia.net}
\end{center}
Se una persona ha più di un nome o di un cognome si usa il primo nome e il  
primo cognome: Carlo Mario Bianchi Rossi studente diventa
\begin{center}
	\printmail{carlo.bianchi.stu@iisperugia.net}
\end{center}
In caso di omonimia viene aggiunto un numero dopo il cognome.
\section{Mail istituzionale e Privacy}
Partiamo da qualche osservazione: la mail è inviolabile, la 
MI è uno strumento di lavoro. L'utente deve impegnarsi a controllarla perché questa permette la comunicazione fra lui e la scuola. 

La MI è 
un dato sensibile, per proteggere la privacy del titolare,  sono quindi 
disponibili delle mail alias~\footcite{Garante2007}\index{Mail!istituzionale!alias} che nascondono l'identità di utenti 
che  hanno rapporti 
con l'esterno dell'organizzazione~\footnote{segreteria@iisperugia.net, pcto@iisperugia.net}. 



Per garantire la privacy dell'utente, la scuola fornendo questo servizio, non può, salvo casi eccezionali, utilizzare 
la mail personale dell'utente.

L'utente non può a sua volta utilizzare la MI per  usi  
personali dato il profilo istituzionale di questa. 

 All'inizio del rapporto con la scuola, l'utente dovrebbe essere informato di questo 
in modo da non portare a fraintendimenti in futuro.

Il Garante della privacy~\footcite{Garante2007}\index{Garante della privacy} chiede che 
venga definito un disciplinare, noto agli utenti, che spieghi le procedure con cui vengono la MI. 
L'utente deve conoscere il percorso di vita della sua utenza, creazione, uso, 
recupero eventuale di dati, 
cancellazione~\footcite{Garante2007}~\footcite{Garante2019}. 

L'utente deve 
essere consapevole che 
può recuperare i propri dati trasferendo  la sua utenza. 

La MI permette l'utilizzo all'utente delle varie applicazioni che formano \textenglish{Google Workspace}
\section{Alias}
Un email\index{Mail!alias} è una mail di inoltro che viene aggiunta dall'amministratore ad un utente\footcite{Google2023d} tramite console\index{Console!amministrazione!alias}. Tale mail non può essere condivisa ne è un account.
L'amministratore può creare fino a trenta alias diversi per utente.
\begin{center}
	
\begin{tabular}{lll}
	\toprule 
	Titolo&Nome&Mail Alias\\
	\midrule
	Animatore digitale&
	Danilo Ardillo&
	animatore@iisperugia.net\\
	Bullismo Madonna Alta&
	Matteo Castellini&
	bullismomalta@iisperugia.net\\
	Bullismo Olmo&
	Amelia Sabatino&
	bullismoolmo@iisperugia.net\\
	Corso Serale&
	Daniela Strona&
	serale@iispeugia.net\\
	Disciplinare&
	Alessia Nunzi&
	disciplinare@iisperugia.net\\
	Educazione Civica&
	Angela Longo&
	civica@iisperugia.net\\
	Erasmus+&
	Lucia Cozzolino&
	erasmus@iisperugia.net\\
	Invalsi&
	Marisa Pettirossi&
	invalsi@iisperugia.net\\
	PCTO manutenzione&
	Marco Biccheri&
	pctomanutenzione@iisperugia.net\\
	PCTO moda&
	Paola Viscuso&
	pctomoda@iisperugia.net\\
	PCTO Olmo&
	Stefano Rossi Ciucci&
	pctoolmo@iisperugia.net\\
	PCTO Pascal&
	Annalisa Cicioni&
	pctopascal@iisperugia.net\\
	Pon e Formazione&
	Danilo Ardillo&
	pon@iisperugia.net\\
	Referente corso elettronica&
	Pasquale Costanzo&
	corsoe@iisperugia.net\\
	Referente corso gestione acque&
	Moretti Orietta&
	corsog@iisperugia.net\\
	Referente corso meccanica&
	Francesco Dottori&
	corsot@iisperugia.net\\
	Referente corso moda&
	Paola Viscuso&
	corsoa@iisperugia.net\\
	Referente corso servizi commerciali&
	Antonella Pesce&
	corsob@iisperugia.net\\
	Referente Dsa/altroBes&
	Simonetta Tini&
	dsabes@iispeugia.net\\
	Referente Inclusione&
	Annalisa Federici&
	rinclusione@iisperugia.net\\
	Sostegno Madonna Alta&
	Annalisa Federici&
	sostegnomalta@iisperugia.net\\
	Sostegno Olmo&
	Stefania Pedini&
	sostegnoolomo@iisperugia.net\\
	Sostegno Piscille&
	Elisa Castellani&
	sostegnopiscille@iisperugia.net\\
	Supervisore Google Workspace &
	Danilo Ardillo&
	supervisore@iisperugia.net\\
	Supervisore Google Workspace Madonna Alta&
	Annalisa Cicioni&
	supervisoremalta@iisperugia.net\\
	Supervisore Google Workspace Olmo&
	Bryan Balingit&
	supervisoreolmo@iisperugia.net\\
	Supervisore Google Workspace Piscille&
	Claudio Duchi&
	supervisorepiscille@iisperugia.net\\
	Viaggi di istruzione e visite didattiche&
	Daniela Strona&
	viaggi@iispeugia.net\\
	\bottomrule
\end{tabular}
\captionof{table}{Alias attivi}
\end{center}
