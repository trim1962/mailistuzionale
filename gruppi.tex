% !TeX spellcheck = it_IT
% !TeX root = UsoMail.tex
\chapter{Gruppi}
%\printpartialtoc
\section{Caratteristiche e struttura}
Ogni gruppo ha:
\begin{itemize}
	\item Nome	
	\item Email gruppo@iisperugia.net
	\item Descrizione
	\item Proprietario gli Amministratori\footnote{Salvo casi particolari}
	\item Tipo di accesso
\end{itemize}

\begin{center}
\begin{tabular}{p{3.2cm}p{1.5cm}p{1.5cm}p{1.5cm}p{1.5cm}p{1.5cm}}%{lccccc}
\bottomrule
Impostazioni di accesso	& Proprietari del Gruppo &  Gestore del Gruppo &
Membri del Gruppo &
Intera Organizzazione &
Esterno\\
\midrule
Chi può contattare i proprietari del gruppo	&  \surd &  \surd &  \surd & & \\[1ex]
\midrule
Chi può visualizzare le conversazioni	&  \surd &  \surd &  \surd & & \\[1ex]
\midrule
Chi può  pubblicare post		&  \surd &  \surd &  \surd & \surd &  \\[1ex]
\midrule
Chi può visualizzare i membri	&  \surd &  \surd &  \surd & \surd &  \\
\midrule
Chi può gestire i membri		&  \surd &  \surd  \\
\bottomrule
\end{tabular}
	\captionof{table}{Impostazioni di accesso gruppi}
\end{center}
Non sono ammessi membri esterni all'organizzazione\footnote{Per inserire elementi non appartenenti alla scuola  bisogna intervenire da console, entrare nel gruppo e inserire l'utente esterno}

Per popolare un gruppo\index{Gruppo!popolare gruppo} è possibile utilizzare da console le UA. Prima si crea il gruppo, definendo privacy etc. Poi, da console, si Seleziona la UA, quindi si selezionano gli utenti. Dal menu Altre opzioni --> Aggiungi utenti selezionati ai gruppi. Ad ogni gruppo è possibile assegnare un calendario.

Altro modo per aggiungere utenti è utilizzare  i gruppi nidificati\footcite{Google2023a}\index{Gruppo!nidificato}. Un gruppo è nidificato se è membro di un altro gruppo. Per esempio il gruppo Coordinatori Cavour Marconi Pascal ha come membri i gruppi: Coordinatori Madonna Alta, Coordinatori Olmo e Coordinatori Piscille. Il prossimo gruppo Docenti A.S. 23/24 avrà come membri i gruppi dipartimento. Ottenendo una migliore gestione di questo gruppo.

\begin{center}
	\begin{tabularx}{\linewidth}{>{\setlength\hsize{.8\hsize}}X>{\setlength\hsize{\hsize}\setlength\linewidth{\hsize}}X}
\toprule	
Regola&	Descrizione\\
\midrule
I membri dei gruppi secondari non sono membri diretti dei gruppi principali	&
I membri dei gruppi secondari non appartengono ai gruppi principali, di conseguenza:
\begin{itemize}
	\item Non possono lasciare il gruppo principale.
	\item Non compaiono come membri del gruppo principale.
	\item Non possono cambiare le loro impostazioni di iscrizione nel gruppo principale.
\end{itemize}\\
I membri dei gruppi secondari ereditano alcune autorizzazioni dai gruppi principali&
Se un file di Documenti Google viene condiviso con un gruppo principale, anche i membri del gruppo secondario potranno accedervi.\\
I membri dei gruppi secondari possono pubblicare post nel gruppo principale&
Se a tutti i membri del gruppo principale è consentito pubblicare post, anche i membri del gruppo secondario potranno pubblicare post nel gruppo principale.\\
I membri dei gruppi secondari possono ricevere messaggi pubblicati nei gruppi principali&Se vuoi che i membri dei gruppi secondari ricevano i messaggi pubblicati nel gruppo principale, imposta l'autorizzazione Chi può pubblicare per il gruppo secondario su Tutti i membri dell'organizzazione. In caso contrario, i membri del gruppo secondario potrebbero non ricevere i messaggi pubblicati nel gruppo principale.\\
\bottomrule
\end{tabularx}
	\captionof{table}{Regole gruppi nidificati}
\end{center}
\section{Insegnanti di Classroom}
Un gruppo fondamentale per il funzionamento di Classroom\index{Classroom} è Insegnanti di Classroom\footcite{Google2023}\index{Gruppo!Insegnanti di Classroom}. Chi è membro effettivo o in sospeso può creare delle Classroom\index{Classroom}. 

L'amministratore deve verificare chi è iscritto o ha chiesto l'iscrizione a questo gruppo. La sua descrizione riporta:

	Il gruppo include tutti gli insegnanti di Classroom\index{Classroom} del tuo dominio. Gli utenti che si presentano come insegnanti sono aggiunti come membri in attesa fino ad approvazione. I membri del gruppo hanno privilegi aggiuntivi che puoi gestire nella Console di amministrazione.

Il gruppo non deve essere mai cancellato.
\section{Gruppi per il funzionamento della scuola }
Sono gruppi creati con le regole precedentemente definite e non ammettono, salvo forzature, membri esterni della scuola. Al momento l'unico membro esterno ammesso è \printmail{PGIS03300A@istruzione.it}\index{PGIS03300A}
\begin{center}
	\begin{tabular}{ll}
		\toprule
Coordinatori Cavour Marconi Pascal*&
coordinatoricmp@iisperugia.net \\
Coordinatori Cavour Marconi Pascal*&
coordinatoricmp@iisperugia.net  \\
Coordinatori dipartimenti*&
gcd@iisperugia.net\\
Coordinatori Madonna Alta AS 22-23*&
cmalta@iisperugia.net\\
Coordinatori Olmo AS 22-23*&
colmo@iisperugia.net\\
Coordinatori Piscille AS 22-23*&
cpiscille@iisperugia.net\\
Dipartimento discipline economiche giuridiche*&
dip\_discipline\_economiche\_giuridiche@iisperugia.net\\
Dipartimento discipline scientifiche*&
dip\_discipline\_scientifiche@iisperugia.net\\
Dipartimento discipline umanistiche*&
dip\_discipline\_umanistiche@iisperugia.net\\
Dipartimento elettrici elettronici*&
dip\_elettrici\_elettronici@iisperugia.net\\
Dipartimento lingue*&
dip\_lingue@iisperugia.net\\
Dipartimento meccanica*&
dip\_meccanica@iisperugia.net\\
Dipartimento moda*&
dip\_moda@iisperugia.net\\
Dipartimento religione*&
dip\_religione@iisperugia.net\\
Dipartimento scienze motorie*&
dip\_scienze\_motorie@iisperugia.net\\
Dipartimento sostegno*&
dip\_sostegno@iisperugia.net\\
Docenti as 22/23*&
docenti\_22-23@iisperugia.net\\
Funzioni Strumentali*&
fs@iisperugia.net\\
Gruppo autovalutazione di istituto&
niv@iisperugia.net\\
Gruppo di lavoro per l'inclusione*&
gli@iisperugia.net\\
Gruppo lavoro PCTO*&
glpcto@iisperugia.net\\
Gruppo tecnico UDA-PFI&
gtup@iisperugia.net\\
Orientamento in ingresso*&
orienta@iisperugia.net\\
Referenti Covid 19&
referenti.covid@iisperugia.net\\
Staff*&
staff@iisperugia.net\\
Team Antibullismo &
antibullismo@iisperugia.net\\
Team Innovazione*&
innovazione@iisperugia.net\\
Team prevenzione dispersione*&
teamdispgdl@iisperugia.net\\
Tutor PCTO Madonna Alta*&
tutormalta@iisperugia.net\\
Tutor PCTO Olmo*&
tutorolmo@iisperugia.net\\
Tutor PCTO Piscille* &
tutorpiscille@iisperugia.net\\
Tutor PCTO* Scuola&
tutorscuola@iisperugia.net\\
\midrule
\end{tabular}
\captionof{table}{Gruppi attivi}
\end{center}
\section{Gruppi speciali}
Sono gruppi che non rispettano le regole precedenti e che in alcuni casi, permettono utenti esterni alla scuola.
\subsection{recipient}
Questo gruppo ha praticamente solo utenti esterni alla scuola. Ha come membro \printmail{PGIS03300A@istruzione.it}\index{PGIS03300A} e autorizza membri esterni alla scuola.
\begin{center}
	\begin{tabular}{p{3.2cm}p{1.5cm}p{1.5cm}p{1.5cm}p{1.5cm}p{1.5cm}}%{lccccc}
		\bottomrule
		Impostazioni di accesso	& Proprietari del Gruppo &  Gestore del Gruppo &
		Membri del Gruppo &
		Intera Organizzazione &
		Esterno\\
		\midrule
		Chi può contattare i proprietari del gruppo	&  \surd &  \surd &  \surd &  \surd& \surd \\[1ex]
		\midrule
		Chi può visualizzare le conversazioni	&  \surd &  \surd &  \surd & & \\[1ex]
		\midrule
		Chi può  pubblicare post		&  \surd &  \surd &  \surd & &  \\[1ex]
		\midrule
		Chi può visualizzare i membri	&  \surd &  \surd &  \surd & \surd &  \\
		\midrule
		Chi può gestire i membri		&  \surd &  \surd  \\
		\bottomrule
	\end{tabular}
	\captionof{table}{Impostazioni di accesso gruppo recipient}
\end{center}
