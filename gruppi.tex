% !TeX spellcheck = it_IT
% !TeX root = UsoMail.tex
\chapter{Gruppi}
\section{Caratteristiche e struttura}
Ogni gruppo ha:
\begin{itemize}
	\item Nome	
	\item Email gruppo@iisperugia.net
	\item Descrizione
	\item Proprietario Gli Amministratori\footnote{Salvo casi particolari}
	\item Tipo di accesso
\end{itemize}

\begin{center}
\begin{tabular}{lccccc}
\bottomrule
\multicolumn{1}{p{3.2cm}}{Impostazioni di accesso}	& \multicolumn{1}{p{1.6cm}}{Proprietari del Gruppo} &  \multicolumn{1}{p{1.6cm}}{Gestore del Gruppo} &
\multicolumn{1}{p{1.6cm}}{Membri del Gruppo} &
\multicolumn{1}{p{1.6cm}}{Intera Organizzazione} &
\multicolumn{1}{p{1.6cm}}{Esterno}\\
\midrule
Chi può contattare i proprietari del gruppo	&  \surd &  \surd &  \surd & & \\[1ex]
\midrule
Chi può visualizzare le conversazioni	&  \surd &  \surd &  \surd & & \\[1ex]
\midrule
Chi può  pubblicare post		&  \surd &  \surd &  \surd & \surd &  \\[1ex]
\midrule
Chi può visualizzare i membri	&  \surd &  \surd &  \surd & \surd &  \\
\midrule
Chi può gestire i membri		&  \surd &  \surd  \\
\bottomrule
\end{tabular}
	\captionof{table}{Impostazioni di accesso}
\end{center}
Non sono ammessi membri esterni all'organizzazione\footnote{Per inserire membri esterni bisogna utilizzare la console e entrare nel gruppo e inserire l'esterno}

Un gruppo fondamentale per il funzionamento di Classroom è Insegnanti di Classroom\footcite{Google2023}\index{Gruppo!Insegnanti di Classroom}. Chi è membro o membro in sospeso può creare delle Classroom. 

L'amministratore deve verificare chi è iscritto o ha chiesto l'iscrizione a questo gruppo. La descrizione del gruppo riporta:

	Il gruppo include tutti gli insegnanti di Classroom del tuo dominio. Gli utenti che si presentano come insegnanti sono aggiunti come membri in attesa fino ad approvazione. I membri del gruppo hanno privilegi aggiuntivi che puoi gestire nella Console di amministrazione.

Il gruppo non deve essere mai cancellato.

\begin{center}
	\begin{tabular}{ll}
		\toprule
Coordinatori Cavour Marconi Pascal*&
coordinatoricmp@iisperugia.net \\
Coordinatori Cavour Marconi Pascal*&
coordinatoricmp@iisperugia.net  \\
Coordinatori dipartimenti*&
gcd@iisperugia.net\\
Coordinatori Madonna Alta AS 22-23*&
cmalta@iisperugia.net\\
Coordinatori Olmo AS 22-23*&
colmo@iisperugia.net\\
Coordinatori Piscille AS 22-23*&
cpiscille@iisperugia.net\\
Dipartimento discipline economiche giuridiche*&
dip\_discipline\_economiche\_giuridiche@iisperugia.net\\
Dipartimento discipline scientifiche*&
dip\_discipline\_scientifiche@iisperugia.net\\
Dipartimento discipline umanistiche*&
dip\_discipline\_umanistiche@iisperugia.net\\
Dipartimento elettrici elettronici*&
dip\_elettrici\_elettronici@iisperugia.net\\
Dipartimento lingue*&
dip\_lingue@iisperugia.net\\
Dipartimento meccanica*&
dip\_meccanica@iisperugia.net\\
Dipartimento moda*&
dip\_moda@iisperugia.net\\
Dipartimento religione*&
dip\_religione@iisperugia.net\\
Dipartimento scienze motorie*&
dip\_scienze\_motorie@iisperugia.net\\
Dipartimento sostegno*&
dip\_sostegno@iisperugia.net\\
Docenti as 22/23*&
docenti\_22-23@iisperugia.net\\
Funzioni Strumentali*&
fs@iisperugia.net\\
Gruppo autovalutazione di istituto&
niv@iisperugia.net\\
Gruppo di lavoro per l'inclusione&
gli@iisperugia.net\\
Gruppo lavoro PCTO&
glpcto@iisperugia.net\\
Gruppo tecnico UDA-PFI&
gtup@iisperugia.net\\
Orientamento in ingresso&
orienta@iisperugia.net\\
Referenti Covid 19&
referenti.covid@iisperugia.net\\
Staff&
staff@iisperugia.net\\
Team Antibullismo &
antibullismo@iisperugia.net\\
Team Innovazione&
innovazione@iisperugia.net\\
Team prevenzione dispersione&
teamdispgdl@iisperugia.net\\
Tutor PCTO Madonna Alta&
tutormalta@iisperugia.net\\
Tutor PCTO Olmo&
tutorolmo@iisperugia.net\\
Tutor PCTO Piscille &
tutorpiscille@iisperugia.net\\
Tutor PCTO Scuola&
tutorscuola@iisperugia.net\\
\midrule
\end{tabular}
\captionof{table}{Gruppi attivi}
\end{center}


