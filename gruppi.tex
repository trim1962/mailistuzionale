% !TeX spellcheck = it_IT
% !TeX root = UsoMail.tex
\chapter{Gruppi}
%\printpartialtoc
\section{Caratteristiche e struttura}
Un gruppo è un insieme di utenti che hanno una caratteristica in comune.
Ogni gruppo ha:
\begin{itemize}
	\item Nome	
	\item Email gruppo@iisperugia.net
	\item Descrizione
	\item Proprietario 
	\item Gestore
	\item Tipo di accesso
\end{itemize}
La funzione primaria di un gruppo è lo scambio di mail fra i membri. Un utente scrive alla mail del gruppo e questa viene consegnata a tutti gli altri appartenenti. 

Un utente  è membro\index{Utente!menbro} se appartiene ad un gruppo o è in sospeso\index{Utente!sospeso} se ha fatto domanda di appartenere ad un gruppo.

I gruppi possono essere creati o da console\index{Console!amministrazione} o tramite l'applicazione gruppi.
I due metodi non sono equivalenti\footcite{Google2023f}. Se un gruppo viene creato dall'applicazione non può essere utilizzato per configurazione\index{Gruppo!gruppo di configurazione} di funzionalità o servizi.
 
La mail del gruppo rappresenta gli utenti di quel gruppo e permette azioni collettive. Un esempio di azione collettiva è l'inserimento di alunni in una Classroom\index{Classroom}. Se gli alunni appartengono allo stesso gruppo classe sarà sufficiente inserire la sola mail del gruppo per caricarli contemporaneamente nella Classroom. Purtroppo,  Classroom\index{Classroom}, ha un bug e permette di caricare solo dieci docenti con questo metodo. 

L'uso del gruppo facilita la gestione di Drive condivisi\index{Drive!condiviso}, Calendari, documenti etc. cioè in tutte quelle situazioni in cui è necessario condividere risorse con molte persone. L'inserimento della sola mail del gruppo risparmia la digitazione di molte mail.

La tabella seguente rappresenta le regole con cui sono stati creati gran parte dei gruppi della nostra scuola. Per la privacy del gruppo, bisogna fare attenzione a non impostare "Chi può visualizzare le conversazioni" su Intera Organizzazione. Altrimenti si corre il rischio concreto che membri dell'organizzazione, leggano le mail interne del gruppo.
\begin{center}
\begin{tabular}{p{3.2cm}p{1.5cm}p{1.5cm}p{1.5cm}p{1.5cm}p{1.5cm}}%{lccccc}
\bottomrule
Impostazioni di accesso	& Proprietari del Gruppo &  Gestore del Gruppo &
Membri del Gruppo &
Intera Organizzazione &
Esterno\\
\midrule
Chi può contattare i proprietari del gruppo	&  \surd &  \surd &  \surd & & \\[1ex]
\midrule
Chi può visualizzare le conversazioni	&  \surd &  \surd &  \surd & & \\[1ex]
\midrule
Chi può  pubblicare post		&  \surd &  \surd &  \surd & \surd &  \\[1ex]
\midrule
Chi può visualizzare i membri	&  \surd &  \surd &  \surd & \surd &  \\
\midrule
Chi può gestire i membri		&  \surd &  \surd  \\
\bottomrule
\end{tabular}
	\captionof{table}{Impostazioni di accesso gruppi}
\end{center}
Non sono ammessi membri esterni all'organizzazione\footnote{Per inserire elementi non appartenenti alla scuola  bisogna intervenire da console\index{Console!amministrazione}, entrare nel gruppo e inserire l'utente esterno}

L'inserimento di utenti avviene da console\index{Console!amministrazione}, dall'applicazione gruppi\index{Applicazione!gruppi}. Questo inserimento di utenti avviene sia alla creazione del gruppo che successivamente. Tale iscrizione dipende dalle regole  ed è permessa solo agli utenti proprietari e/o gestori.

Per popolare un gruppo\index{Gruppo!popolare gruppo} è possibile utilizzare, da console, le UA\index{Unità!Organizzativa}. Si  procede in questo modo: prima si crea il gruppo, definendo privacy etc, poi, da console, si trova la UA che vogliamo inserire nel gruppo, quindi si selezionano gli utenti della UA. Dal menu Altre opzioni --> Aggiungi utenti selezionati ai gruppi. 

 Ad ogni gruppo è possibile assegnare un calendario o un drive condiviso\index{Drive!condiviso!gruppo}

Un altro modo per aggiungere utenti è utilizzando i gruppi nidificati\footcite{Google2023a}\index{Gruppo!nidificato}. Un gruppo è nidificato se è membro di un altro gruppo. Per esempio il gruppo Coordinatori Cavour Marconi Pascal ha come membri i gruppi: Coordinatori Madonna Alta, Coordinatori Olmo e Coordinatori Piscille. Il prossimo gruppo Docenti A.S. 23/24 avrà come membri i gruppi dipartimento. Ottenendo una migliore gestione di questo gruppo.
\todo{spiegare o modificare}
\todo{Inserire inserimenti tramite cvs}
\begin{center}
	\begin{tabularx}{\linewidth}{>{\setlength\hsize{.8\hsize}}X>{\setlength\hsize{\hsize}\setlength\linewidth{\hsize}}X}
\toprule	
Regola&	Descrizione\\
\midrule
I membri dei gruppi secondari non sono membri diretti dei gruppi principali	&
I membri dei gruppi secondari non appartengono ai gruppi principali, di conseguenza:
\begin{itemize}
	\item Non possono lasciare il gruppo principale.
	\item Non compaiono come membri del gruppo principale.
	\item Non possono cambiare le loro impostazioni di iscrizione nel gruppo principale.
\end{itemize}\\
I membri dei gruppi secondari ereditano alcune autorizzazioni dai gruppi principali&
Se un file di Documenti Google viene condiviso con un gruppo principale, anche i membri del gruppo secondario potranno accedervi.\\
I membri dei gruppi secondari possono pubblicare post nel gruppo principale&
Se a tutti i membri del gruppo principale è consentito pubblicare post, anche i membri del gruppo secondario potranno pubblicare post nel gruppo principale.\\
I membri dei gruppi secondari possono ricevere messaggi pubblicati nei gruppi principali&Se vuoi che i membri dei gruppi secondari ricevano i messaggi pubblicati nel gruppo principale, imposta l'autorizzazione Chi può pubblicare per il gruppo secondario su Tutti i membri dell'organizzazione. In caso contrario, i membri del gruppo secondario potrebbero non ricevere i messaggi pubblicati nel gruppo principale.\\
\bottomrule
\end{tabularx}
	\captionof{table}{Regole gruppi nidificati}
\end{center}
\section{Regole mail}
\todo{spiegare chiaramente le regole per le mail}
\section{Insegnanti di Classroom}
Un gruppo fondamentale per il funzionamento di Classroom\index{Classroom} è Insegnanti di Classroom\footcite{Google2023}\index{Gruppo!Insegnanti di Classroom}. Chi è membro effettivo o in sospeso di questo  gruppo può creare le Classroom\index{Classroom}. 

L'amministratore deve verificare chi è iscritto o ha chiesto l'iscrizione a questo gruppo. La sua descrizione riporta:

	Il gruppo include tutti gli insegnanti di Classroom\index{Classroom} del tuo dominio. Gli utenti che si presentano come insegnanti sono aggiunti come membri in attesa fino ad approvazione. I membri del gruppo hanno privilegi aggiuntivi che puoi gestire nella Console di amministrazione\index{Console!amministrazione}.

Il gruppo non deve essere mai cancellato.
\section{Gruppi per il funzionamento della scuola }
Sono gruppi creati con le regole precedentemente definite e non ammettono, salvo forzature, membri esterni della scuola. Al momento l'unico membro esterno ammesso è \printmail{PGIS03300A@istruzione.it}\index{PGIS03300A}
\begin{center}
	\begin{tabular}{ll}
		\toprule
Nome gruppo&Mail\\
\midrule
Coordinatori Cavour Marconi Pascal*&
coordinatoricmp@iisperugia.net  \\
Coordinatori dipartimenti*&
gcd@iisperugia.net\\
Coordinatori Madonna Alta AS 22-23*&
cmalta@iisperugia.net\\
Coordinatori Olmo AS 22-23*&
colmo@iisperugia.net\\
Coordinatori Piscille AS 22-23*&
cpiscille@iisperugia.net\\
Dipartimento discipline economiche giuridiche*&
dip\_discipline\_economiche\_giuridiche@iisperugia.net\\
Dipartimento discipline scientifiche*&
dip\_discipline\_scientifiche@iisperugia.net\\
Dipartimento discipline umanistiche*&
dip\_discipline\_umanistiche@iisperugia.net\\
Dipartimento elettrici elettronici*&
dip\_elettrici\_elettronici@iisperugia.net\\
Dipartimento lingue*&
dip\_lingue@iisperugia.net\\
Dipartimento meccanica*&
dip\_meccanica@iisperugia.net\\
Dipartimento moda*&
dip\_moda@iisperugia.net\\
Dipartimento religione*&
dip\_religione@iisperugia.net\\
Dipartimento scienze motorie*&
dip\_scienze\_motorie@iisperugia.net\\
Dipartimento sostegno*&
dip\_sostegno@iisperugia.net\\
Docenti as 22/23*&
docenti\_22-23@iisperugia.net\\
Funzioni Strumentali*&
fs@iisperugia.net\\
Gruppo autovalutazione di istituto&
niv@iisperugia.net\\
Gruppo di lavoro per l'inclusione*&
gli@iisperugia.net\\
Gruppo lavoro PCTO*&
glpcto@iisperugia.net\\
Gruppo tecnico UDA-PFI&
gtup@iisperugia.net\\
Orientamento in ingresso*&
orienta@iisperugia.net\\
Referenti Covid 19&
referenti.covid@iisperugia.net\\
Staff*&
staff@iisperugia.net\\
Team Antibullismo &
antibullismo@iisperugia.net\\
Team Innovazione*&
innovazione@iisperugia.net\\
Team prevenzione dispersione*&
teamdispgdl@iisperugia.net\\
Tutor PCTO Madonna Alta*&
tutormalta@iisperugia.net\\
Tutor PCTO Olmo*&
tutorolmo@iisperugia.net\\
Tutor PCTO Piscille* &
tutorpiscille@iisperugia.net\\
Tutor PCTO* Scuola&
tutorscuola@iisperugia.net\\
Segreteria Amministrativa&amministrativa@iisperugia.net	\\
Segreteria Didattica&didattica@iisperugia.net	\\
Segreteria Personale&personale@iisperugia.net	\\
Segreterie Personale&segreteria@iisperugia.net	\\
\midrule
\end{tabular}
\captionof{table}{Gruppi attivi}
\end{center}
\section{Dipartimenti}
La nostra scuola ha dieci dipartimenti che riuniscono le classi di concorso
\begin{center}
	\begin{tabular}{lc}
\toprule
Nome	& Classi di concorso \\
\midrule
Dipartimento discipline economiche giuridiche*	& \begin{tabular}{cccc}
		A010& A066 & A018 &A054  \\
		B022	& A045 &A046  & A041 \\		
	\end{tabular} \\
	\midrule
Dipartimento discipline scientifiche*	&\begin{tabular}{cccc}
		A034	& A020 &A026  & A050 \\
		B012	&  &  &  \\
	\end{tabular}\\
	\midrule
Dipartimento discipline umanistiche*	&\begin{tabular}{cc}
	A012	& A021  \\
\end{tabular}\\
	\midrule
Dipartimento elettrici elettronici*&\begin{tabular}{cc}
	A040	& B015  \\
\end{tabular}\\
	\midrule
Dipartimento lingue*&\begin{tabular}{cc}
	AB24	& AA24  \\
\end{tabular}\\
	\midrule
Dipartimento meccanica*&\begin{tabular}{cc}
	A042	& B017  \\
\end{tabular}\\
	\midrule
Dipartimento moda*&\begin{tabular}{ccc}
	A017	& A044&B018  \\
\end{tabular}\\
	\midrule
Dipartimento religione*&\\
	\midrule
Dipartimento scienze motorie*&A048\\
	\midrule
Dipartimento sostegno*&\\
\bottomrule
\end{tabular}
	\captionof{table}{Composizione gruppi dipartimentali}
\end{center}

\begin{center}
	\begin{tabular}{p{3.2cm}p{1.5cm}p{1.5cm}p{1.5cm}p{1.5cm}p{1.5cm}}%{lccccc}
		\bottomrule
		Impostazioni di accesso	& Proprietari del Gruppo &  Gestore del Gruppo &
		Membri del Gruppo &
		Intera Organizzazione &
		Esterno\\
		\midrule
		Chi può contattare i proprietari del gruppo	&  \surd &  \surd &  \surd &&  \\[1ex]
		\midrule
		Chi può visualizzare le conversazioni	&  \surd &  \surd &  \surd & & \\[1ex]
		\midrule
		Chi può  pubblicare post		&  \surd &  \surd &  \surd &\surd &  \\[1ex]
		\midrule
		Chi può visualizzare i membri	&  \surd &  \surd &  \surd & &  \\
		\midrule
		Chi può gestire i membri		&  \surd &  \surd  \\
		\bottomrule
	\end{tabular}
	\captionof{table}{Impostazioni di accesso gruppo didattica}
\end{center}
\section{Gruppi speciali}
Sono gruppi che non rispettano le regole precedenti e che in alcuni casi, permettono utenti esterni alla scuola.
\subsection{Didattica}
Fanno parte di questo gruppo i membri della segreteria didattica
\begin{center}
	\begin{tabular}{p{3.2cm}p{1.5cm}p{1.5cm}p{1.5cm}p{1.5cm}p{1.5cm}}%{lccccc}
		\bottomrule
		Impostazioni di accesso	& Proprietari del Gruppo &  Gestore del Gruppo &
		Membri del Gruppo &
		Intera Organizzazione &
		Esterno\\
		\midrule
		Chi può contattare i proprietari del gruppo	&  \surd &  \surd &  \surd &&  \\[1ex]
		\midrule
		Chi può visualizzare le conversazioni	&  \surd &  \surd &  \surd & & \\[1ex]
		\midrule
		Chi può  pubblicare post		&  \surd &  \surd &  \surd &\surd &  \\[1ex]
		\midrule
		Chi può visualizzare i membri	&  \surd &  \surd &  \surd & &  \\
		\midrule
		Chi può gestire i membri		&  \surd &  \surd  \\
		\bottomrule
	\end{tabular}
	\captionof{table}{Impostazioni di accesso gruppi dipartimentali}
\end{center}
\subsection{Amministrativa}
Fanno parte di questo gruppo i membri della segreteria amministrativa.
\begin{center}
	\begin{tabular}{p{3.2cm}p{1.5cm}p{1.5cm}p{1.5cm}p{1.5cm}p{1.5cm}}%{lccccc}
		\bottomrule
		Impostazioni di accesso	& Proprietari del Gruppo &  Gestore del Gruppo &
		Membri del Gruppo &
		Intera Organizzazione &
		Esterno\\
		\midrule
		Chi può contattare i proprietari del gruppo	&  \surd &  \surd &  \surd &&  \\[1ex]
		\midrule
		Chi può visualizzare le conversazioni	&  \surd &  \surd &  \surd & & \\[1ex]
		\midrule
		Chi può  pubblicare post		&  \surd &  \surd &  \surd &\surd &  \\[1ex]
		\midrule
		Chi può visualizzare i membri	&  \surd &  \surd &  \surd & &  \\
		\midrule
		Chi può gestire i membri		&  \surd &  \surd  \\
		\bottomrule
	\end{tabular}
	\captionof{table}{Impostazioni di accesso gruppo amministrativa}
\end{center}
\subsection{Personale}
Fanno parte di questo gruppo i membri della segreteria personale.
\begin{center}
	\begin{tabular}{p{3.2cm}p{1.5cm}p{1.5cm}p{1.5cm}p{1.5cm}p{1.5cm}}%{lccccc}
		\bottomrule
		Impostazioni di accesso	& Proprietari del Gruppo &  Gestore del Gruppo &
		Membri del Gruppo &
		Intera Organizzazione &
		Esterno\\
		\midrule
		Chi può contattare i proprietari del gruppo	&  \surd &  \surd &  \surd &&  \\[1ex]
		\midrule
		Chi può visualizzare le conversazioni	&  \surd &  \surd &  \surd & & \\[1ex]
		\midrule
		Chi può  pubblicare post		&  \surd &  \surd &  \surd &\surd &  \\[1ex]
		\midrule
		Chi può visualizzare i membri	&  \surd &  \surd &  \surd & &  \\
		\midrule
		Chi può gestire i membri		&  \surd &  \surd  \\
		\bottomrule
	\end{tabular}
	\captionof{table}{Impostazioni di accesso gruppo personale}
\end{center}
\subsection{Segreterie}
Fanno parte di questo gruppo i membri delle segreterie della scuola. Il gruppo è un gruppo nidificato\index{Gruppo!nidificato}
\begin{center}
	\begin{tabular}{p{3.2cm}p{1.5cm}p{1.5cm}p{1.5cm}p{1.5cm}p{1.5cm}}%{lccccc}
		\bottomrule
		Impostazioni di accesso	& Proprietari del Gruppo &  Gestore del Gruppo &
		Membri del Gruppo &
		Intera Organizzazione &
		Esterno\\
		\midrule
		Chi può contattare i proprietari del gruppo	&  \surd &  \surd &  \surd &&  \\[1ex]
		\midrule
		Chi può visualizzare le conversazioni	&  \surd &  \surd &  \surd & & \\[1ex]
		\midrule
		Chi può  pubblicare post		&  \surd &  \surd &  \surd &\surd &  \\[1ex]
		\midrule
		Chi può visualizzare i membri	&  \surd &  \surd &  \surd & &  \\
		\midrule
		Chi può gestire i membri		&  \surd &  \surd  \\
		\bottomrule
	\end{tabular}
	\captionof{table}{Impostazioni di accesso gruppo segreteria}
\end{center}
\subsection{Recipient}
Questo gruppo ha praticamente solo utenti esterni alla scuola. Ha come membro \printmail{PGIS03300A@istruzione.it}\index{PGIS03300A} e autorizza membri esterni alla scuola.
\begin{center}
	\begin{tabular}{p{3.2cm}p{1.5cm}p{1.5cm}p{1.5cm}p{1.5cm}p{1.5cm}}%{lccccc}
		\bottomrule
		Impostazioni di accesso	& Proprietari del Gruppo &  Gestore del Gruppo &
		Membri del Gruppo &
		Intera Organizzazione &
		Esterno\\
		\midrule
		Chi può contattare i proprietari del gruppo	&  \surd &  \surd &  \surd &  \surd& \surd \\[1ex]
		\midrule
		Chi può visualizzare le conversazioni	&  \surd &  \surd &  \surd & & \\[1ex]
		\midrule
		Chi può  pubblicare post		&  \surd &  \surd &  \surd & &  \\[1ex]
		\midrule
		Chi può visualizzare i membri	&  \surd &  \surd &  \surd & \surd &  \\
		\midrule
		Chi può gestire i membri		&  \surd &  \surd  \\
		\bottomrule
	\end{tabular}
	\captionof{table}{Impostazioni di accesso gruppo recipient}
\end{center}
\subsection{Coordinatori Piscille, Madonna Alta, Olmo}
Ognuno di questi gruppi ha un proprio calendario. Di ognuno di questi gruppi fanno parte il DS e i Coordinatori della scuola. Gli utenti possono\footcite{Google2023g}:
\begin{itemize}
	\item Trovare i dettagli di tutti gli eventi, inclusi quelli privati.
	\item Aggiungere e modificare eventi.
	\item Ripristinare o eliminare definitivamente gli eventi dal cestino del calendario.
	\item Trovare le impostazioni di fuso orario del calendario.
	\item Effettuare l'iscrizione agli avvisi via email quando gli eventi vengono creati, modificati, annullati, ricevono una risposta o sono imminenti.
\end{itemize}

I gruppi possono accedere al Drive condiviso Coordinatori di Classe.
\section{Ruoli utenti}
Ogni gruppo\footcite{Google2023e} ha 
\begin{itemize}
	\item Proprietario\index{Gruppo!proprietario}. Crea o cancella il gruppo, gestisce le iscrizioni e i messaggi 
	\item Gestore\index{Gruppo!gestore} gestisce le iscrizioni e i messaggi
	\item Utente\index{Gruppo!utente}
\end{itemize}
Gli amministratori di sistema\index{Amministratore!Super amministratore} sono proprietari di ogni gruppo che viene creato. Attualmente i gruppi hanno come proprietari effettivi gli Admin. Tale impostazione deve essere modificata individuando per ogni gruppo un proprietario appartenete al gruppo.